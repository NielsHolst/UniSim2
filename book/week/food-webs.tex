\documentclass [a4paper, 11pt, openany]  {memoir}
\usepackage{amsmath}

\begin{document}

\section{Multiple predator-prey functional response}
We will use the Gutierrez-Baumg{\"a}rtner functional response for one prey and one predator as a basis to derive a many prey-many predators functional response model. The predation outcome, or supply ($\Delta S$; kg prey), is calculated from current predator demand ($\Delta D$; kg prey), the search efficiency of the predator ($s$; kg predator per kg prey) and prey density ($X$; kg prey):

\begin{equation}
\Delta S = \Delta D \left[1-exp\left(\frac{-sX}{\Delta D}\right)\right].
\end{equation}
The demand is a product of the demand rate ($d$, kg prey per kg predator per h), predator density ($Y$; kg predator) and the time step ($\Delta t$; h):

\[ \Delta D = dY\Delta t .\]
The functional response (eq.1) forms a saturation curve defined by its initial slope ($s$) at low prey density and its saturation ($\Delta D$) at high prey density:

\[  \Delta S \rightarrow sX\ \text{for}\ sX/\Delta D  \rightarrow 0 \]
\[  \Delta S \rightarrow \Delta D\ \text{for}\ sX/\Delta D \rightarrow \infty .\]
This means that at low prey density, the functional response (eq. 1) is equivalent to the Lotka-Volterra functional response which is linear at all prey densities.

Logically, the search efficiency ($s$) should increase with the duration of the time step ($\Delta t$). Moreover, the inequality $s \leq 1$ should be obeyed since $\Delta S \rightarrow sX$ at low prey density; the predator cannot kill more than the present prey density. Hence we get

\begin{equation}
s = 1 - exp(-\alpha \Delta t),
\end{equation}
which has the limits
\[  s \rightarrow \alpha\Delta t\ \text{for}\  \alpha\Delta t \rightarrow 0 \]
\[  s \rightarrow 1\ \text{for}\  \alpha\Delta t \rightarrow \infty ,\]
where $\alpha$ denotes the attack rate.

To extend eq. 1 we consider multiple prey ($X_i$) attacked by multiple predators ($Y_j$) with demands $D_j$. The search rates are defined specifically for each predator-prey interaction in an interaction matrix with entries $s_{ij}$.

First we define the total supply over all prey for the $j$'th predator ($\Delta S_{*j}$):

\begin{equation}
\Delta S_{*j} = \Delta D_j \left[1-exp\left(\frac{-\sum_i s_{ij} X_i}{\Delta D_j}\right)\right]
\end{equation}
\[ \Delta D_j = d_jY_j\Delta t ,\]
which accumulates the predation process as $\sum_i s_{ij} X_i$, whereby the prey most easily attacked  by the predator (large $s_{ij}$) or having the largest density ($X_i$) will contribute most to the resulting supply ($\Delta S_{*j}$).

Next the supply is split among the predator's prey in proportion to the $s_{ij} X_i$ term:

\begin{equation}
\Delta S_{ij} = \frac{s_{ij}X_i}{\sum_i s_{ij}X_i} \Delta S_{\ast j}.
\end{equation}
After having calculated the supplies for all predators according to eq. 3, the total kill of each prey($\Delta S_{i*}$) is found by summation across all predators:

\begin{equation}
\Delta S_{i*} = \sum_j \Delta S_{ij}
\end{equation}

We did not yet define the search efficiencies ($s_{ij}$) but return to that now, as we must make sure that the total predation on a prey does not exceed its density, i.e. the inequality $\Delta S_{i*} \leq X_i$ must hold. This we  ensure by forcing the sum of all search efficiences for each predator not to exceed 1, i.e. $s_{i*} \leq 1$, where

\[ s_{i*} = \sum_j s_{ij} \]
We define a first approximation of search efficiency ($\widetilde{s}_{ij}$) on par with eq. 2:

\begin{equation}
\widetilde{s}_{ij} = 1 - exp(-\alpha_{ij} Y_j \Delta t)
\end{equation}
\[ \widetilde{s}_{i*} = \sum_j \widetilde{s}_{ij} \]
The search efficiencies ($s_{ij}$) are equal to $\widetilde{s}_{ij}$ if the sum pertaining to the prey ($\widetilde{s}_{i*}$) does not exceed one; otherwise the $\widetilde{s}_{ij}$ values are scaled down to yield $s_{i*} = 1$:

\begin{equation}
s_{ij}=\begin{cases}
               \widetilde{s}_{ij}\ \text{for}\ \widetilde{s}_{i*} < 1\\
               \widetilde{s}_{ij} / \widetilde{s}_{i*}\ \text{otherwise}
            \end{cases}
\end{equation}




\end{document}
