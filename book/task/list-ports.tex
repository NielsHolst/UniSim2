To find out what the purpose of a class is and to get a complete list of its
input and output ports, use the \code{help} command together with the class
name, for instance, to see the impressive list of \code{Calendar} ports:
\begin{usdialog}
> help Calendar
Calendar keeps check on date, time and %\brk%other sun-earth relations

Input:
.latitude        double    Latitude (degrees)
.longitude       double    Longitiude (degrees)
.timeZone        int       Time zone (h)
.initialDateTime QDateTime %\brk%Date and time when calendar starts
.timeStep        int       %\brk%Time step in units of timeUnit
.timeUnit        char      %\brk%Unit of time step (y,d,h,m,s
.sample          int       %\brk%The frequency at which output is sampled

Output:
.date            QDate     Current date
.time            QTime     %\brk%Current time of the day
.trueSolarTime   QTime     %\brk%Current true solar time of the day
.dateTime        QDateTime %\brk%Current date and time
.timeStepSecs    double    %\brk%Time step duration (s)
.timeStepDays    double    %\brk%Time step duration (d)
.totalTimeSteps  int       %\brk%Total number of time steps since calendar%\brk%was reset
.totalTime       int       %\brk%Total time since calendar was reset; in units%\brk%of timeUnit
.totalDays       double    %\brk%Total time since calendar was reset (d)
.dayOfYear       int       Julian day
.dayLength       double    %\brk%Astronomic day length
.sinb            double    %\brk%Sine of solar height over the horizon
.azimuth         double    %\brk%The compass direction of the sun relative to%\brk%north [-180;180]
.sunrise         QTime     %\brk%Time of sunrise
.sunset          QTime     %\brk%Time of sunset
.solarConstant   double    %\brk%The irradiation at the top of the atmosphere%\brk%(MJ/m^2/d)
.angot           double    %\brk%The irradiation at Earth surface under optimal%\brk%atmospheric conditions (MJ/m^2/d)
\end{usdialog}
Note that class names (like all other names in \CPP, R and box scripts) are case-sensitive.

For each port its name and type is given. Here we see \code{int}, \code{double}, \code{QDate} and \code{QDateTime}. The type of a port is defined in the \CPP\ code for the box class.