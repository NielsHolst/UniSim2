In some cases you my have a box which depends on a variable number of inputs. Let's say a \code{Competition} box needs the density and competition coefficient of every \code{Plant} box present in the model:

Simulation {
  Competition {
  }
  Box community {
    SoilNitrogen {
    }
    Plant aster {
      .a = 0.8
    }
    Plant poppy {
      .a = 0.2
    }
    Plant violet {
      .a = 0.1
    }
  }
}

Here, the \code{Plant} has been coded with \code{a} as an input for the competition coefficient. Moreover, it maintains an output called \code{density} (\CPP\ code not shown).

We need to code the \code{Competition} class to look up the \code{Plant} boxes present inside the \code{community} class and to build a vector referring to those \code{Plant} boxes. There are two solutions to this. Either keep a vector of pointers to all \code{Plant} objects, or keep one vector of pointers to \code{a} values and another one with pointers to \code{density} values. Which to choose depends mostly on what you find most convenient. The second method will execute slight faster.

\subsection{Vector to \protect\code{Box}objects}
You declare the vector in the \code{private} section of your class:

#include <QVector>
#include <base/box>

namespace meadow {
	class Community : public base:Box {
	public:
	Community();
	void initialize();
	void update();
	private:
	// Inputs
	QVector<double> a, density;
	// Outputs
	double aAll;
	};
}

Here, \code{aValues} is a vector of pointers to variables of type \code{double}. The \code{const} keyword signifies that you intention is to read these values not to change them. In fact, the compiler will disallow code that changes any of the values pointed to.

The right place to fill the \code{aValues} vector is in the \code{initialize} method, which will be called only once after you have executed the \uscom{run} command:

Input(a).imports("community/*<Plant>[a]");
Input(density).imports("community/*<Plant>[density]");

Here, we apply one of the advanced features of box paths. The \code{<Plant>} notation means 'of class \code<Plant>'. In effect, we will find any box that is a child of the box named \code{community} and in that find the port named \code{a} (line 1) or \code{density} (line 2).

int n1 = a.size(), 
    n2 = density.size();
if (n1 != n2) 
	ThrowException("Vector sizes for a and density do not match").value1(n1).value2(n2).context(this);
aAll = 0;
double sum = 0;
for (int i=0; i<n1; ++i) {
	aAll += a[i]*density[i];
	sum += density[i];
}
aAll /= sum;
