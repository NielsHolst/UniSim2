You can use any text editor to edit box scripts. However, as described in \iref{ch:install-text-editor} you should install one of the text editors for which \US\ provides integration: Atom (for Linux, OS X and Windows) or \NPP\ (for Windows, recommended).

The \US\ \rcom{save grammmar} command will update the grammar file of your installed text editor(s) (Atom, \NPP\ or both) (\iref{ch:install-text-editor}). The command will take the currently available plug-ins, with their defined box classes and ports, and use them as a basis to construct the grammar file. This means that, if you are developping your own plug-in, your names for box classes and ports will only get highligthed in the text editor after you have run the \rcom{save grammmar} command. 

Note that, if the text editor was running while you issued the \rcom{save grammmar} command, you will need to close and re-open the text editor for the saved grammar to take effect.

If any of the box objects in a box script creates ports dynamically (in the \code{amend} step, see \iref{ch:computations}), these ports will only appear in the grammar file if you first load the box script file and then save the grammar file.

Blind ports, (\ie\ ports that you define in the box script with the plus (\code{+}) syntax will never be saved to the grammar file and, hence, cannot be highlighted in the text editor.

