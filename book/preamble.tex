%
% Load packages
%

\usepackage{amsmath}
\usepackage[export]{adjustbox}
\usepackage{caption}
\usepackage{subcaption}
\usepackage{color}
\usepackage{afterpage}
\usepackage{xcolor}
\usepackage{xcolor-solarized} % http://ethanschoonover.com/solarized
\usepackage{fontawesome}
% \usepackage{gensymb}
\usepackage{graphicx}
%\usepackage{helvet}
%\usepackage{inconsolata}
\usepackage{charter}
\usepackage{keystroke}
\usepackage{lmodern}
\usepackage{natbib}
\usepackage{paralist}
\usepackage{placeins}
\usepackage[T1]{fontenc}
\usepackage{siunitx}
\usepackage[normalem]{ulem}
\usepackage{underscore}
\usepackage{url}

\usepackage{varioref}

\usepackage{listings}

\usepackage{tikz}
\usetikzlibrary{shapes, shadows, arrows}
\usetikzlibrary{trees}

% Load after fonts
\usepackage{microtype}

% These should be the last two packages to be loaded:
\usepackage {hyperref}
\hypersetup{
    colorlinks = true,
    linkcolor = {solarized-orange},
    citecolor = {solarized-magenta}
}
\usepackage{cleveref}

% Further setup instructions
\crefname{equation}{equation}{equations}
\crefname{subequation}{equation}{equations}
\newtheorem{theorem}{Theorem}[section]
% \numberwithin{equation}{section}



\setlength\beforechapskip{-\baselineskip}

%
% Define commands
%
\newcommand{\front}[1]{\label{ch:#1}\input{front/#1}}
\newcommand{\week}[1]{\label{ch:#1}\input{week/#1}}
\newcommand{\task}[1]{\label{ch:#1}\input{task/#1}}
\newcommand{\uschapter}[1]
  {\chapter{#1}\label{us:#1}\input{reference/us/#1}}
\newcommand{\xuschapter}[1]
  {\chapter{#1}\label{us:#1}}
\newcommand{\cppchapter}[1]
  {\label{cpp:#1}\input{reference/cpp/#1}}
\newcommand{\xchapter}[1]{\chapter{#1 (missing)}}

\newcommand{\xtask}[1]{\label{ch:#1}To be written.}
\newcommand{\xmyref}[1]{\label{ch:#1}To be written.}

\newcommand{\iref}[1]{\textit{\autoref{#1}}}

\newcommand{\mytilde}{\raise.17ex\hbox{$\scriptstyle\mathtt{\sim}$}} 
\newcommand{\US}{Universal Simulator}
\newcommand{\USI}{Universal Simulator Installer}
\newcommand{\USS}{Universal Simulator Support}
\newcommand{\CPP}{\nolinebreak[4]{C\texttt{++}}}
\newcommand{\NPP}{Note\nolinebreak[4]{pad\texttt{++}}}
\newcommand{\GB}{Gutierrez-Baumg{\"a}rtner}
\newcommand{\cf}{\textit{cf.}}
\newcommand{\ie}{\textit{i.e.}}
\newcommand{\eg}{\textit{e.g.,}} 
\newcommand{\etc}{\textit{etc.}}
\newcommand{\vs}{\textit{vs.}}
\newcommand{\etal}{\textit{et al.}}
\newcommand{\aka}{\textit{aka}}
\newcommand{\viceversa}{\textit{vice versa}}
\newcommand{\brk}{\newline \-\ \quad \(\hookrightarrow\)}
\newcommand{\ushome}{\faHome}
\newcommand{\inputfolder}{\faSignIn}
\newcommand{\outputfolder}{\faSignOut}
\newcommand{\devhome}{\faWrench}
\newcommand{\xaxis}{$x$-axis}
\newcommand{\yaxis}{$y$-axis}
\newcommand{\matr}[1]{\mathbf{#1}} % matrix notation; use as $\matr{A}$


\newcommand{\inputfolderexplained}{{\protect\inputfolder}\footnote{\protect\inputfolder\ is your \protect\US\ input folder (see \protect\iref{commands:set-folder})}}

\newcommand{\filenameexplained}[1]{{\protect\filename{\protect\inputfolder/#1}}\footnote{\protect\inputfolder\ is your \protect\US\ input folder (see \protect\iref{commands:set-folder})}}

\newcommand{\ushomeexplained}{{\protect\ushome}\footnote{\protect\ushome\ is the user work folder (see \protect\iref{commands:set-folder})}}

\newcommand{\devhomefolderexplained}{{\protect\devhome}\footnote{\protect\devhome\ is the developer's work folder (see \protect\iref{commands:set-folder})}}

\newcommand{\devhomeexplained}[1]{{\protect\filename{\protect\devhome/#1}}\footnote{\protect\devhome\ is the developer's work folder (see \protect\iref{commands:set-folder})}}

\newcommand{\outputfolderexplained}{\protect\outputfolder \footnote{\protect\outputfolder\ is your \protect\US\ output folder (see \protect\iref{commands:set-folder})}}

\newcommand{\filename}[1]{\hbox{\texttt{#1}}}
\newcommand{\codenobox}[1]{\texttt{#1}}
\newcommand{\code}[1]{\hbox{\texttt{#1}}}
\newcommand{\concept}[1]{\textbf{#1}}
\newcommand{\gui}[1]{\textbf{#1}}
\newcommand{\litt}[1]{\uwave{#1}}
\newcommand{\rcom}[1]{\texttt{#1}}
\newcommand{\uscom}[1]{\texttt{#1}}

\newcommand{\commandsection}[1]{\section{#1}\label{commands:#1}}
\newcommand{\commandsectionlabel}[2]{\section{#1}\label{commands:#2}}


% Keys
\newcommand{\keyfont}[1]{\texttt{#1}}
\newcommand{\autofillkey}{\keyfont{<Ctrl>+<Space>}}
\newcommand{\downkey}{\keyfont{<Down>}}
\newcommand{\pastekey}{\keyfont{<Ctrl>+V} or \keyfont{<Right-click>} menu}
\newcommand{\upkey}{\keyfont{<Up>}}
\newcommand{\esckey}{\keyfont{<Esc>}}
\newcommand{\returnkey}{\keyfont{<Return>}}

\newcommand{\todo}[1]{\uwave{#1}}
\newcommand{\usinfo}[1]{\textcolor{blue}{#1}}

  
%
% Define listings
%

\lstset{basicstyle=\ttfamily\small, 
         tabsize=2, 
         xleftmargin=17pt,
         numberstyle=\sffamily\color{solarized-orange}\scriptsize,
         escapeinside=\%\%
        }

\lstdefinelanguage{prompted}
	{morecomment=[l]{>}
	}

\lstnewenvironment{boxscript}
  {\lstset{escapeinside=\%\%}}
  {}

\lstnewenvironment{rscript}
  {\lstset{escapeinside=\%\%}}
  {}

\lstnewenvironment{cpp}
  {\lstset{escapeinside=\%\%}}
  {}

\lstnewenvironment{usoutput}{}{}

\lstnewenvironment{textfile}{}{}

\lstnewenvironment{usdialog}
	{ \lstset{basicstyle=\ttfamily\small\color{blue},
		      commentstyle=\ttfamily\small\color{black},
		      language=prompted}
	}
	{}

\lstnewenvironment{rdialog}
	{ \lstset{basicstyle=\ttfamily\small\color{black},
		      commentstyle=\ttfamily\small\color{blue},
		      language=prompted,
          escapeinside=\%\%}
	}
	{}

 \lstnewenvironment{userror}
	{ \lstset{basicstyle=\ttfamily\small\color{red},
		      commentstyle=\ttfamily\small\color{black},
		      language=prompted,
          escapeinside=\%\%}
	}
	{}


\lstnewenvironment{us}{}{}
  
\lstnewenvironment{uscmd}{}{}

